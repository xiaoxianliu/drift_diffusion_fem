\documentclass{article}
\usepackage{amsmath}
\usepackage{amssymb}

\title{Numerical Solution to Drift-Diffusion Model for Organic Semiconductors}
\author{Xiaoxian Liu}
\date{\today}

\newcommand{\grad}{\overrightarrow{\nabla}}
\newcommand{\Jn}{\overrightarrow{J}_n}
\newcommand{\Jp}{\overrightarrow{J}_p}
\newcommand{\Ju}{\overrightarrow{J}_u}


\begin{document}
\maketitle


\section{Drift-diffusion model}
 \subsection{Model description}
 The list of unknowns is given below. Let
 \begin{itemize}
  \item $\{\psi, n, p, u\}$ be the known electric potential, electron density, hole density, and exciton density, 
 respectively;
  \item $\Jn, \Jp, \Ju$ be the flux of $n, p, u$;
 \end{itemize}
 
 (Need to fill in geometric information of domain...)\\
 
 Then the {\it Drift-Diffusion equations} for the organic semiconductor is
 \begin{align}
  &-\grad\cdot\left( \epsilon_r(x) \epsilon_0 \grad \psi \right) = q(p-n) 
    &\\
  &\grad\cdot \Jn = \delta(\Gamma)\, h\left( k(\grad\psi) u - \gamma np \right), 
    & \Jn = -D_n\grad n + n\mu_n\grad\psi \\
  &\grad\cdot \Jp = \delta(\Gamma)\, h\left( k(\grad\psi) u - \gamma np \right), 
    & \Jp = -D_p\grad p - p\mu_p\grad\psi \\
  &\grad\cdot \Ju = Q(x) - \frac{u}{\tau_u} - \delta(\Gamma)\, h\left( k(\grad\psi) - \gamma np \right),
    & \Ju = -D_u\grad u
 \end{align}
 where
 \begin{itemize}
  \item $q$ is the unit charge of an electron; $\epsilon_0$ and $\epsilon_r$ are the vacumm electrical permittivity 
    and relative permittivity, respectively
  \item $k(\grad \psi)$ is the dissociation coefficient of exciton into free electron-hole carrier pairs
  \item $\gamma$ is the recombination coefficient of electron-hole pairs into excitons
  \item $Q(x)$ is the photo-generation rate of exciton; (the only generation rate from external source, i.e. sunlight)
  \item $\tau_u$ is the lifetime of exciton decaying into ground state
  \item $\delta(\Gamma)$ is the Delta distribution concentrating on the interface $\Gamma$
  \item $h$ is the interface widths
 \end{itemize}
 
 (Need to add boundary conditions herein...)

 
 

  
 
 
 
 \subsection{Scaled model for numerical computation}
 For numerical simulation, we need to introduce some constant values for each dimension. We let
 \begin{itemize}
  \item Universal constants 
     \begin{itemize}
	\item Unit charge: $q = 1.6 \times 10^{-19} $ C
	\item Boltzman constant: $k_B = 1.38 \times 10^{-23} \, m^2\cdot kg\cdot s^{-2}\cdot K^{-1}$
	\item Vacumm electric permittivity: $\epsilon_0 = 8.854 \times 10^{-12} \, F\cdot m^{-1}$
     \end{itemize}
  \item Physical settings of device
      \begin{itemize}
	\item Temperature: $T = 300 K$, room temperature
	\item Device length: $L = 200nm = 2\times10^{-7}m$
	\item Number density: $N=1\times 10^{20}\,m^{-3}$
	\item Interface width: $h=2nm = 2\times10^{-9} m$
	\item Typical mobility value: $\mu_0 = 1\times 10^{-9} m^2\cdot V^{-1}\cdot s^{-1}$
      \end{itemize}
  \item Derived quantities
      \begin{itemize}
	\item Thermal potential: $U_T = k_B T / q$
	\item Typical time scale: $\tau = \frac{L^2}{\mu_0 U_T}$, i.e. the time for a particle with diffusivity 
	  $\mu_0 UT$ to diffuse across the device with length $L$
	\item $\lambda^2 = \frac{\epsilon_0 U_T}{qNL^2} = \frac{L_{Debye}^2}{L^2}$, 
	  where $L_{Debye}^2 = \frac{\epsilon_0 U_T}{q N}$
      \end{itemize}
 \end{itemize}
 Furthermore, we assume the Einstein relationship holds for each species, i.e.
 \begin{align}
  D_i = \mu_i U_T
 \end{align}
 where ''i'' can be either $n$, $p$, or $u$. \\

 To derive the scaled equations, we use these quantities above to scale the physical quantities in original 
 drift-diffusion equations. For instance,
 $\psi = U_T \psi_s$, $\grad = \frac{1}{L} \grad_s$, where subscription ''s'' indicates it's a scaled 
 quantity/operator. In what's coming, we shall omite subscription ''s'' for simplicity. Hence, the {\it\bf scaled 
 drift-diffuion equations} are
 \begin{align}
 &-\lambda^2 \grad \cdot \left( \epsilon_r \grad\psi\right) = p - n,
   & \\
 &\grad \cdot \Jn = \delta(\Gamma) h\left( k(\grad\psi) - \gamma np \right)
   & \Jn = -\mu_n(\grad\psi) \left(\grad n - n\grad\psi \right)\\
 &\grad \cdot \Jp = \delta(\Gamma) h\left( k(\grad\psi) - \gamma np \right) 
   & \Jp = -\mu_p(\grad\psi) \left(\grad p + n\grad\psi \right)\\
 &\grad \cdot \Ju = Q - \frac{u}{\tau_u} - \delta(\Gamma) h\left( k(\grad\psi) - \gamma np \right)
   & \Ju = -\mu_u \grad u
 \end{align}


\section{Numerical Simulation}
 \subsection{Gummel's iteration}
 Gummel's iteration is a solution map of drift-diffusion equation of Gauss-Seidel fashion. Concretely, for the 
 scaled drift-diffusion equations introduced above, we start with an initial guess of solutions $\psi^{(0)}, 
 n^{(0)}, p^{(0)}, u^{(0)}$. For $l = 1,2,...$, given $\psi^{(l-1)}, n^{(l-1)}, p^{(l-1)}, u^{(l-1)}$, 
 the l-th step of Gummel's iteration consists of the following steps:
 \begin{itemize}
  \item Solve 
    $-\grad \left[ \mu_n(\grad\psi^{(l-1)}) \left( \grad n^{(l)} - n^{(l)}\grad \psi^{(l-1)} \right) \right] 
    = \delta(\Gamma) h\left[ k(\grad\psi^{(l-1)})u^{(l-1)} - \gamma n^{(l)}p^{(l-1)} \right]$
    for $n^{(l)}$.
  \item Solve
    $-\grad \left[ \mu_p(\grad\psi^{(l-1)}) \left( \grad p^{(l)} + p^{(l)} \grad \psi^{(l-1)} \right) \right]
    = \delta(\Gamma) h\left[ k(\grad\psi^{(l-1)})u^{(l-1)} - \gamma n^{(l-1)}p^{(l)} \right]$
    for $p^{(l)}$.
  \item Solve
    $-\grad \left( \mu_u \grad u^{(l)} \right) 
    = Q - \frac{u^{(l)}}{\tau_u} - \delta(\Gamma) h\left[ k(\grad\psi^{(l-1)})u^{(l)} - \gamma n^{(l-1)}p^{(l-1)} \right]$
    for $u^{(l)}$
  \item Solve
    $-\lambda^2 \grad \left( \epsilon_r \grad \psi^{(l)} \right)
    = p^{(l)} - n^{(l)}$
    for $\psi^{(l)}$
 \end{itemize}

 \subsection{Scharfetter-Gummel discretization for convection-diffusion equations}
 
\end{document}
