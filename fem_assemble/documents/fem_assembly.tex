\documentclass{article}
\usepackage{amsmath}

\begin{document}
 \section{Prototype equation}
 \subsection{Geometry}
 Let $\Omega$ be a bounded 2-dimensional region. Let $\Gamma$ be a 1-dimensional curve that devides $\Omega$ into two 
 subregions, denoted as $\Omega_1$ and $\Omega_2$.  Let $\partial \Omega_D$ and $\partial \Omega_N$ denote its 
 Dirichlet and Neumann boundaries, respectively. \\
 
 \subsection{Diffusion-reaction equation for excitons}
 Consider the following equation defined on $\Omega$:
 \begin{align}
  -\nabla \left( a(x) \nabla u \right) + c(x) u &= f(x) 
 \end{align}
 and boundary conditions are:
 \begin{align}
  u = u_D && \forall x \in \Omega_D\\
  \frac{\partial u}{\partial n} = g_N && \forall x \in \Omega_N
 \end{align}
 
 In particular, $u$ is not assumed to have continuous ''flux'' across $\Gamma=\overline{\Omega}_1 \cap 
 \overline{\Omega}_2$, i.e.
 \begin{align}
  &u_1 = u_2 = u_\Gamma && \forall x \in \Gamma \\
  &\frac{\partial u_1}{\partial n_{12}} - \frac{\partial u_2}{\partial n_{12}} = -ku_\Gamma
	  && \forall x \in \Gamma
 \end{align}
 The physical meaning here is that, if $u$ is the density of excitons, then near the interface $\Gamma$, $u$ has a 
 drainage rate of ''$ku$''.\\
 
 To use Finite Element Method, one needs to write down the weak form. Let $v$ be a test function on $\Omega$, then
 \begin{align}
  \int_\Omega a(x) \nabla u \cdot \nabla v\, dx 
  +\int_\Omega c(x)uv\,dx
  +\int_\Gamma kuv\,ds 
  &= \int_\Omega fv dx
 \end{align}
 In FEM, the region $\Omega$ is decomposed into the union of disjoint triangles. Both $u$ and $v$ are approximated by 
 piecewise linear polynomials. 


 \section{Linear system assembly}
 In this section, we write down the detailed formula of assembling coefficient matrix. In particular, the 3 integrals 
 above are replaced by the matrices $A$, $C$, and $D$.

 \subsection{Quadrature rule: 1D and 2D}
 Consider an integral over a triangular region with vertices \{$v_0$,$v_1$,$v_2$\}
 \begin{align}
  I &=\int_T f\,dx
 \end{align}
 Then first-order quadrature rule is
 \begin{align}
  I &\approx |T| * (f_0 + f_1 + f_2) / 3
 \end{align}
 where $f_i$ denotes the function value $f(x_i, y_i)$.\\
 
 We then consider an integral over a segment, for example an edge $e$ of a triangle with end nodes being \{ $v_0$, 
 $v_1$\}. And we have 1st-order quadrature rule
 \begin{align}
  I &= \int_e f \, ds\\
   &\approx |e|\, \frac{f_0 + f_1}{2}
  \end{align}

 These quadrature rules are exact for linear polynomials and are used for assembling all the matrices below.

 \subsection{Matrices and vectors}
 \subsubsection{Matrix A}
 Let $\phi_i$ be the linear basis Lagrange polynomial with $1$ on the i-th node and $0$ on all other nodes. Then
 \begin{align}
  A_{ij} &= \int_\Omega a(x)\nabla \phi_i \nabla\phi_j \, dx
 \end{align}
 
 In practice, this is done by looping through all triangular elements $T$'s. For each $T$, one identify its vertices 
 $i$, $j$, $k$, and calculate the contribution of $T$ to all 9 entries in A: $A_{\alpha \beta}$ where $\alpha$ and 
 $\beta$ can be either $i$, $j$, or $k$.\\
 
 For example, assume triangle $T$ has vertices 0,1,and 2. We denote the coordinates of these vertices by 
 $\overrightarrow{r}_0$, $\overrightarrow{r}_1$ and $\overrightarrow{r}_2$. We compute 2 quantities: $A_{00}(T)$
 and $A_{01}(T)$. Other cases can be obtained by switching subscripts between 0, 1, and 2.
 
 \begin{enumerate}
  \item $A_{00}(T)$\\
  First, note by a linear transformation, one can transform $v_0$, $v_1$ and $v_2$ easily to a reference triangle with 
  vertices (0,0), (1,0), and (0,1). In this way, we compute
  \begin{align}
   \nabla \phi_0 \cdot \nabla \phi_0 
   &= \frac{\left| \overrightarrow{r}_2 - \overrightarrow{r}_1 \right|^2}
	   {4\,|T|^2}
  \end{align}
  And hence,
  \begin{align}
   \int_T a(x) \nabla \phi_0 \cdot \nabla \phi_1 \, dx 
   &= |T| \cdot \frac{\left( a_0 + a_1 + a_2 \right)}{3} 
	      \cdot \frac{\left| \overrightarrow{r}_2 - \overrightarrow{r}_1 \right|^2}{4\,|T|^2} \\
   &= \frac{a_0 + a_1 + a_2}{12 \, |T|}\,\left| \overrightarrow{r}_2 - \overrightarrow{r}_1 \right|^2
  \end{align}


  \item $A_{01}(T)$\\
  Similar to $A_{01}(T)$, we again make use of the reference triangle and obtained
  \begin{align}
   \nabla \phi_0 \cdot \nabla \phi_1 
   & = -\frac{      \left( \overrightarrow{r}_2 - \overrightarrow{r}_0 \right) 
              \cdot \left( \overrightarrow{r}_2 - \overrightarrow{r}_1 \right)}
             {4\,|T|^2}
  \end{align}
  and thus
  \begin{align}
   \int_T a(x) \nabla \phi_0 \cdot \nabla \phi_1 \, dx
   &= - \frac{a_0 + a_1 + a_2}{12 \, |T|}
        \,      \left( \overrightarrow{r}_2 - \overrightarrow{r}_0 \right) 
          \cdot \left( \overrightarrow{r}_2 - \overrightarrow{r}_1 \right)
  \end{align}
  Note the ''-'' sign on the right-hand side and symmetry in the formulation.

 \end{enumerate}

 \subsubsection{Matrix C}
 Applying the simple quadrature rule, we conclude the matrix $C$ is diagonal, i.e. $A_{ij}=0$ if $i \neq j$.
 We then compute the contribution to $A_{ii}$ from element $T$ if $v_i$ is a vertex belonging to $T$. 
 \begin{align}
  A_{ii}(T) &= \int_T c(x) \phi_i \phi_i \, dx\\
            &= |T| \; \frac{c_i}{3}
 \end{align}

 \subsubsection{Matrix D}
 Similar structure takes place in $D$. In fact $D$ is sparser than $C$, for it's only non-zero entries corresponds to
 the nodes on the interface $\Gamma$ defined above. Furthermore, applying a 1st-order quadrature rule for line 
 integral, we have 
 \begin{align}
  \int_e k\phi_i \phi_j \, ds 
  &= \delta_{ij}\; |e| \frac{k_i}{2}
 \end{align}
 where $|e|$ is the length of edge $e$.
 
 \subsubsection{Right-hand side vector}
 \begin{align}
  \int_T f(x) \phi_i \, dx
  &= |T| \frac{f_i}{3}
 \end{align}
 
 
 
 
 \subsection{Dirichlet boundary conditions}
 Finally, we identify the nodes on Dirichlet boundaries. We modify the linear system defined above on both sides
 (assuming $v_i$ is a node on Dirichlet boundaries):
 \begin{itemize}
  \item Left side:\\
  We re-write the i-th row as a unit vector with i-th entry being 1 and
  0's elsewhere.
  \item Right side:\\
  Corresponding to the modification to the coefficient matrix on the left side, we change the i-th entry of right-hand 
  side vector to be the Dirichlet boundary value $u_D(v_i)$
 \end{itemize}


\end{document}
